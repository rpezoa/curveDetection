\documentclass{beamer}


\setbeamertemplate{background canvas}[vertical shading][bottom=red!10,top=blue!10]

\usetheme{Warsaw}
%\usecolortheme{crane}
\usefonttheme[onlysmall]{structurebold}
\setbeamerfont{title}{shape=\itshape,family=\rmfamily}
%\setbeamercolor{title}{fg=red!80!black}
\usepackage{tikz}
\usepackage{pgf}
\usepackage{amsmath,amssymb}
\usepackage{tabularx}
\usepackage{supertabular}
\usepackage[latin1]{inputenc}
\usepackage{gensymb}

\usepackage{colortbl}
\usepackage{listings}
\setbeamercovered{dynamic}
%\usepackage[spanish]{babel}
\usepackage[spanish,english]{babel}
\selectlanguage{english}

\title[Image Processing]{Image Processing}
\author[R. Pezoa]{%
  Raquel~Pezoa}
\institute[Universidad T\'ecnica Federico Santa Mar\'ia]{
  %
  Departamento de Inform\'atica\\
  Universidad T\'ecnica Federico Santa Mar\'ia
}
\date[Abril 2012]{Abril 2012}
\subject{Seminario}



%\pgfdeclareimage[opciones]{nombre de imagen}{nombre del archivo, sin extension}

\pgfdeclareimage[width = 0.5\textwidth]{ihc1}{img/ihc1}
\pgfdeclareimage[width = 0.8\textwidth]{ihc1-det}{img/ihc1-det}
\pgfdeclareimage[width = 0.5\textwidth]{nefroide}{img/nefroide}
\pgfdeclareimage[width = 0.5\textwidth]{mix}{img/mix}
\pgfdeclareimage[width=5cm]{normalForm}{img/normalForm}
\pgfdeclareimage[width=4cm]{fakeImage}{img/fakeImage}
\pgfdeclareimage[width=5cm]{curve}{img/curve}
\pgfdeclareimage[width=4.5cm]{ellipse}{img/ellipse}
%nombre de ambientes en español
%\newtheorem{theorem2}{Teorema}
%\newtheorem{example2}{Ejemplo}
%\newtheorem{definition2}{Definición} 


\begin{document}
  
  \frame{\titlepage}
  
% \section*{Outline} 
% \begin{frame}
%   \frametitle{Outline}
%   \tableofcontents
% \end{frame}

\begin{frame}
  \frametitle{Image Segmentation IHC-images}

\begin{figure}
\centering
\pgfuseimage{ihc1}
\end{figure}

\end{frame}

\begin{frame}
  \frametitle{Image Segmentation IHC-images}

\begin{figure}
\centering
\pgfuseimage{ihc1-det}
\end{figure}
\textcolor{blue}{How can we complete (generate a curve) for the membrane that we do not see?}
\end{frame}


\begin{frame}
  \frametitle{Hough Transform}
  \begin{itemize}
  \item Let's remember (a bit) the HT:
\begin{itemize}
  \item The representation of a line as the normal-form,
   considers a point $(x,y)$ as a function of an angle normal
    to the line, passing through the origin of the image:
    \begin{equation}
      \rho = x  \cos\theta + y \sin\theta
    \end{equation}
\end{itemize}
    \end{itemize}
\end{frame}


\begin{frame}
  \frametitle{Normal-form Representation}
\begin{columns}
\column{0.5\textwidth}
  \begin{figure}
    \pgfuseimage{normalForm}
    \caption{Polar consideration of a line.}
    \label{fig:polarLine}
  \end{figure}
\column{0.5\textwidth}

  $$\rho = x  \cos\theta + y \sin\theta$$


$\theta$: angle of the line normal to the line in an image

$\rho$: length between the origin and the point where the lines
intersect.

This gives a form of the HT for lines known as the \emph{polar HT for
  lines.}
\end{columns}
\end{frame}



\begin{frame}

  \begin{columns}
    \column{0.5\textwidth}
    \begin{figure}
      \pgfuseimage{fakeImage}
      \caption{Simple line.}
      \label{fig:fakeLine}
    \end{figure}
    \column{0.5\textwidth}
    \begin{figure}
      \pgfuseimage{curve}
      \caption{Hough Transform of the line.}
      \label{fig:HTPolar}
    \end{figure}
\end{columns}
\end{frame}


\begin{frame}
  \frametitle{HT for Circles}
  \begin{itemize}
  \item The parametric form of a circle is:
    \begin{equation}
      \begin{array}{c c}
      x=x_{0}+r\cos{\theta} &y=y_{0}+r\sin{\theta}
      \end{array}
      \label{eq:circle2}
    \end{equation}

  \item The HT    mapping is:
    \begin{equation}
      \begin{array}{c c}
      x_{0}=x-r\cos{\theta} &y_{0}=y-r\sin{\theta}
      \end{array}
    \end{equation}

    \item These equations define the points in the accumulator space
      dependent of the radius $r$.
\end{itemize}
\end{frame}


\begin{frame}[fragile]
\frametitle{Implementation}
  \tiny{\begin{verbatim} 
function [acc] = HTCircle(im,r)
% im is a binary image
[rows,columns]=size(im);
acc=zeros(rows,columns);
for x=1:columns
    for y=1:rows
        if(im(y,x)==1)
            for ang=0:360
                t=(ang*pi)/180;
                x0=round(x-r*cos(t));
                y0=round(y-r*sin(t));
                if(x0 < columns && x0 >0 && y0 <rows && y0>0)
                    acc(y0,x0)=acc(y0,x0)+1; % accumulator is 2D, fixed value of r
                end
            end
        end
    end
end\end{verbatim}}
\textcolor{blue}{Here, only is consider a fixed r!}
\end{frame}


\begin{frame}
  \frametitle{HT for Ellipses}

The parametric functions of the ellipse are:

$$\begin{bmatrix}x(\theta) \\ y(\theta) \end{bmatrix} = \lambda \begin{bmatrix} \cos \rho & \sin \rho \\ -\sin \rho &\cos \rho \end{bmatrix} \begin{bmatrix} a \cos \theta \\ b \sin \theta \end{bmatrix}$$

\end{frame}



\begin{frame}
  \frametitle{HT for Ellipses}
  \begin{itemize}
  \item It is possible to define a mapping between the circle and an
  ellipse by a transformation:

  \begin{equation}
    \begin{bmatrix} x \\ y\end{bmatrix} = \begin{bmatrix} \cos(\rho) &\sin(\rho) \\ -\sin(\rho) & \cos(\rho) \end{bmatrix} \begin{bmatrix} S_{x} x' \\ S_{y}y'\end{bmatrix} + \begin{bmatrix} t_{x} \\ t_{y}\end{bmatrix}
        \label{eq:ellipse}
  \end{equation}
  $(x',y')$: coordinates of he circle

  $\rho$: represents de orientation

  $(S_{x},S_{y})$: scale factor

  $(t_{x},t_{y})$: translation
  \end{itemize}
\end{frame}

\begin{frame}
  \frametitle{HT for Ellipses}

  \begin{itemize}
  \item If we define:
    \begin{equation}
      \begin{array}{c}
        a_{0}=t_{x} \ \ \  a_{x}=S_{x}\cos(\rho) \ \ \  b_{x}=S_{y}\sin(\rho)\\
        
        b_{0}=t_{y} \ \ \  a_{y}=S_{y}\sin(\rho)\ \ \  b_{y}=S_{y}\cos(\rho)
      \end{array}
    \end{equation}

  \item The circle is \emph{deformed} into:

    \begin{equation}
      \begin{array}{c}
        x=a_{0}+a_{x}\cos(\theta)+b_{x}\sin(\theta)\\
        
        y=b_{0}+a_{y}\cos(\theta)+b_{y}\sin(\theta)
      \end{array}
      \label{eq:ellipse2}
    \end{equation}

    Which corresponds to the polar representation of an ellipse.
  \end{itemize}
\end{frame}


\begin{frame}
\frametitle{HT for Ellipses}
\begin{columns}
\column{0.5\textwidth}

\begin{itemize}
  \item The location of the centre of the ellipse is given by:
\begin{equation}
\begin{array}{c}
   a_{0}=x-a_{x}\cos(\theta)+b_{x}\sin(\theta) \\
   b_{0}=x-a_{y}\cos(\theta)+b_{y}\sin(\theta)
   \end{array}
\end{equation}
  %\item The location is dependant of the three parameters
  \item The mapping defines the trace of a hypersurface in
    five-dimensional (5D) space $\rightarrow$ \textcolor{blue}{this space can be very
    large.}
  \end{itemize}

\column{0.5\textwidth}
\begin{figure}
    \pgfuseimage{ellipse}
    \caption{Definition of ellipse axes.}
    \label{fig:ellipse}
  \end{figure}
\end{columns}
\end{frame}



\begin{frame}
\frametitle{HT for other curves?}

\begin{figure}
\centering
\pgfuseimage{nefroide}
\end{figure}

\end{frame}


\begin{frame}
\frametitle{}

\begin{figure}
\centering
\pgfuseimage
\end{figure}

\end{frame}




\begin{frame}
\frametitle{Problem 2}
\begin{itemize}
\item The idea is to match a \emph{model shape} ("ideal world") against a
shape in any image.
\item However, the shape in the image has a different location, orientation
and scale.
\end{itemize}
\end{frame}


\end{document}


